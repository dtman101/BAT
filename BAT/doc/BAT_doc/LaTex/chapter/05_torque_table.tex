\chapter{Standard Tightening Torque (Torque Table)}
The maximum possible preload (load at yield point) of the bolt is influenced by the simultaneously acting 
tension and torsional stresses $\sigma_M$ and $\tau_M$. These two stresses are combined to en equivalent 
uniaxial stress state with the deformation energy theory (von-Mises)
\begin{equation}
    \sigma_{v,M} = \sqrt{\sigma_M^2 + 3 \tau_M^2}
    \label{equ:sig_vM_2}
\end{equation}
where $\sigma_{v,M}$ \equ{equ:sig_vM}, $\sigma_M$ \equ{equ:sig_M} and $\tau_M$ \equ{equ:tau_M} 
are defined in §\ref{sec:stresses} in detail. For the further derivation the following equations apply 
\begin{equation}
    \sigma_M = \frac{F_M}{A_0}, \qquad \tau_M = \frac{M_{th}}{W_P}
    \label{equ:sig_tau}
\end{equation}
where $d_0$ is the diameter of the relevant stress cross section $A_0=\nicefrac{d_0^2 \pi}{4}$ of the
bolt and with the assocaiated polar moment of resistance $W_P$. The torque present at the thread 
interface $M_{th}$ is defined in \equ{equ:M_th}. Equation \equ{equ:sig_vM_2} can be restructured to
\begin{equation}
    \frac{\sigma_{v,M}}{\sigma_M} = \sqrt{1+3\left(\frac{\tau_m}{\sigma_M}\right)^2}
\end{equation}
and if the equivalent uniaxial stress is set to the minimum yield point $\sigma_{v,M}=\sigma_y^{min}$ 
of the bolt material the \emph{permissible assembly normal stress} $\sigma_M^{allow}$ can be defined
\begin{equation}
    \sigma_M^{allow} = \frac{\sigma_y^{min}}{\sqrt{1+3\left(\frac{\tau_m}{\sigma_M}\right)^2}}
\end{equation}
If the two equations \equ{equ:sig_tau} are combined, the factor $\nicefrac{\tau_m}{\sigma_m}$ can be defined 
\begin{equation}
    \frac{\tau_M}{\sigma_M} = \frac{M_{th} A_0}{W_P F_M}
\end{equation}
where for \emph{metric threads} the torque $M_{th}$ can be simplified with the approximation 
$\tan(\varphi + \rho) \approx \tan\varphi + \tan\rho$ to
\begin{subequations}
    \setlength{\jot}{10pt}
    \begin{align}
        M_{th} &= F_M \frac{d_2}{2} \tan(\varphi+\rho) \\
        &= F_M \frac{d_2}{2} \left( \tan\varphi + \tan\rho \right) \\
        &= F_M \frac{d_2}{2} \left( \frac{p}{\pi d_2} + 1.155 \mu_{th} \right)
    \end{align}
\end{subequations}
Generally, for the elastic reagion applies $W_P=W_{P,el}=\nicefrac{d_0^3 \pi}{16}$ and for the
\emph{fully plastic state}\footnotemark[4]\footnotetext[4]{Full plasticity of cross section: constant
torsional stress over the cross section} a correction to the 
polar moment of resistance is applied $W_P=W_{P,pl}=\nicefrac{d_0^3 \pi}{12}$. All these equations can 
be combined to 
\begin{equation}
    \sigma_M^{allow} = \frac{\sigma_y^{min}}{\sqrt{1+3\left[\varkappa \frac{d_2}{d_0}\left(
        \frac{p}{\pi d_2} + 1.155 \mu_{th}^{min} \right) \right]^2}}
    \label{equ:sig_M_allow}
\end{equation}
where 
\begin{subequations}
    \setlength{\jot}{10pt}
    \begin{align*}
        \varkappa&=2 \qquad \text{for elastic region\footnotemark[5]} \qquad W_P=\frac{d_0^3 \pi}{16} \\
        \varkappa&=\frac{2}{3} \qquad \text{for plastic region\footnotemark[5]} \qquad W_P=\frac{d_0^3 \pi}{12}
    \end{align*}
\end{subequations}
\footnotetext[5]{The elastic region is used in ECSS-E-HB-32-23A \cite{ECSS_HB_32_23A} and is more 
conservative than the plastic region, which is used in VDI 2230 \cite{VDI2230_1}.}
The \emph{permissible assembly preload} $F_M^{allow}$ is defined 
\begin{equation}
    F_M^{allow} = \sigma_M^{allow} A_0 \nu
\end{equation}
with the \emph{bolt utilization factor} $\nu$ \equ{equ:nu}
\begin{equation*}
    \nu = \frac{\sigma_{v,M}^{allow}}{\sigma_y^{min}}
\end{equation*}
Frequently, especially in the case of torque-controlled tightening only one proportional utilization 
(normally $90\%$) is permitted in order to exclude the possibility of the yield point being exceeded in
service \cite{VDI2230_1}.
\begin{colbox}{BAT Info:}
    In BAT the equation \equ{equ:sig_M_allow} is used with both options for elastic $\varkappa=2$ and full 
    plastic region $\varkappa=\nicefrac{2}{3}$. Based on the fact that only fully threaded bolts are used 
    in BAT (no shank bolts) the critical cross section is set to stress area $A_0=A_s$ with the 
    corresponding stress diameter $d_s$. . For the 
    analysis of the correct tightening torque $T_A$ based on the permissible assembly preload the
    simplified version of \equ{equ:TA2} is used 
    \begin{equation}
      T_A = F_M^{allow} \left( 0.16 p + 0.58 d_2 \mu_{th} + 
        \frac{\mu_{uh} D_{Km}}{2\sin \nicefrac{\lambda}{2}} \right)
    \end{equation}
    with $D_{Km}=\nicefrac{(d_h+d)}{2}$ ($D_{hole}$ is neglected and the nominal bolt diameter $d$ is 
    used instead for simplicity) and $\mu_{th}=\mu_{uh}=\mu^{min}$ is used, which leads to the limiting 
    torque. 

    In calculating the tightening torque, it is \textbf{always the minimum coefficient of friction 
    which should be used}, assuming a necessary maximum permissible assembly preload \cite{VDI2230_1}.
\end{colbox}